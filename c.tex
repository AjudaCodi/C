\documentclass[a4paper,10pt]{article}
\usepackage[utf8]{inputenc}
\usepackage[catalan]{babel}

% links de la web
\usepackage{hyperref}

\begin{document}
-- voldria testejar guide o manual than c\\
\part{google}
\section{documentation c}
\begin{itemize}
\item \href{http://www.acm.uiuc.edu/webmonkeys/book/c_guide/}{El meu preferit, des del 1997}
\item \href{http://www.gnu.org/software/gnu-c-manual/gnu-c-manual.html}{Manual GNU}
\item \href{http://www.cs.cf.ac.uk/Dave/C_index.html}{aparenta força cutre ...}
\end{itemize}

\section{learn c}
\begin{itemize}
\item \href{http://www.cprogramming.com/}{Massa general pel meu gust}
\item \href{http://c.learncodethehardway.org/}{Python, ruby, C, Regex, SQL...}
\item \href{http://www.cprogramming.com/tutorial/c-tutorial.html}{Tutorial de c, aparentment decent}
\item \href{http://c.learncodethehardway.org/book/}{Nomes semblen problemes}
\item \href{http://www.learn-c.org/}{Pointer Arithmetics, Function Pointers, Ni idea, sembla xulisim}
\item \href{http://www.c4learn.com/c-programming/c-ansi/}{No me le estudiat}
\end{itemize}

\section{how to yield generator in c}
\begin{itemize}
\item \href{http://eli.thegreenplace.net/2012/04/05/implementing-a-generatoryield-in-a-python-c-extension}{simular un yield en c}
\end{itemize}

\section{source code c}
\begin{itemize}
\item \href{http://www.programmingsimplified.com/c-games-and-projects}{Molt de codi per a veure}
\item \href{http://www.programmingsimplified.com/c-program-examples}{Mes codi per a veure}
\item \href{http://www.programiz.com/c-programming/examples/source-code-output}{Cosa basica per a c}
\end{itemize}

\section{Random c}
\begin{itemize}
\item \href{http://www.programmingsimplified.com/c-program-generate-random-numbers}{Molts d'exemples, esta realment interesant}
\end{itemize}

\section{Debugger}
google: como debugar c con gdb\\
gdb terminal color
\begin{itemize}
\item Compile: --debug
\item run: executa el programa
\item start: para al arribar al main
\item b \{num, method\}: break
	\begin{itemize}
	\item watch exp: Quan canvia de valor exp, llavors es para
	\item info {break, watch}: observar tots els breaks que tens
	\item clear breakpint: elimina un break
	\item delei 13: elimina el break del numero que se veu en info
	\end{itemize}
\item tb \{num, method\}: break only one time
\item n: next line
\item s: next line, if is a method, in this
\item p var: show a var
\item q: exit
\item fr: on estasj
\item bt: mostra el cami a fer per arribar on estas (amb les reucrions es perillos)
\item h: help
\item ENTER: like . in vim
\end{itemize}
\subsection{Debug post-mortem}
Analitzar si algun cop ho necessito\\
\$ gdb programa core\\
Sembla ser que el core es força important\\
\$ gdb programa pid\\
Aparentment, util per debugegar quan hi ha interficie gràfica.
\subsection{Valgrind}
Diuen que es un emulador d'execusions. Aixi que se sap que pasa :)
\subsection{links}
\begin{itemize}
\item http://lihuen.linti.unlp.edu.ar/index.php?title=C\%C3\%B3mo\_usar\_GDB
\item http://www.reloco.com.ar/linux/prog/debug.html
\item[util] http://www.eis.uva.es/~fergay/III/enlaces/gdb.html
\item[info vestia] http://ldc.usb.ve/~figueira/cursos/ci3825/taller/material/gdb.html
\item[core info] http://crysol.github.io/recipe/2006-03-25/depurar-un-programa-c-c/\#.VDpgdnVie00
\item[color] http://stackoverflow.com/questions/209534/prettify-my-gdb
\end{itemize}
\end{document}

if(one)
	if(two)
		foo();
	else
		bar();

From this:

if(one)
	if(two)
		foo();
else
	bar();

